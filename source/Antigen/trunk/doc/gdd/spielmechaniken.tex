Viele Mechaniken von \name{} sind denen in anderen Echtzeitstrategiespielen
ähnlich. Sie werden in den folgenden Abschnitten detailliert beschrieben. Dieser
Abschnitt soll dagegen einen Überblick über die ungewöhnlicheren Spielmechaniken
geben.

\subsubsection{Zellteilung und Mutation}

Neue Zellen werden primär durch Zellteilung produziert. Dabei können die
meisten Zellen sich nur reproduzieren, also Klone von sich selbst erschaffen.
Manche Zellen haben aber auch die Fähigkeit, andere Zelltypen gezielt
herzustellen.

Bei der Zellteilung werden die Eigenschaften der sich teilenden Zelle im
Regelfall unverändert auf ihren Klon übertragen. Mit relativ geringer
Wahrscheinlichkeit tritt allerdings eine Mutation auf, d.h. die Werte des
Klons weichen positiv oder negativ von denen des Originals ab.

In der Spielwelt gibt es bestimmte Bereiche -- Mutationsfelder --, in denen
Mutationen mit deutlich größerer Wahrscheinlichkeit auftreten. Zusätzlich
verändern verschiedene Mutationsfelder verschiedene Eigenschaften der Zellen
unterschiedlich stark.

Die durch Mutationsfelder verursachten Werteänderungen streben einem
Normalniveau entgegen, sodass eine Zelle mit insgesamt besseren Werten
wahrscheinlicher verschlechtert als weiter verbessert wird.

\subsubsection{Antigene}

Jedes Bakterium und jedes Virus trägt eines von mehreren Antigenen in sich.
Diese Antigene können durch Riesenfresszellen extrahiert und anschließend
zur Produktion spezialisierter Abwehrzellen verwendet werden.

Bei der Produktion neuer Bakterien und Viren kann mit sehr geringer
Wahrscheinlichkeit das Antigen mutieren, sodass die neu entstandene Zelle
ein anderes Antigen als die produzierende Zelle trägt. In Mutationsfeldern
ist die Wahrscheinlichkeit hierfür wiederum deutlich erhöht.

\subsubsection{Infektion}

Viren können Zellen des Spielers infizieren, sofern ihre Infektionsstärke die
Virenresistenz der zu infizierenden Zelle übersteigt. Mit der Infektion
verliert der Spieler die Kontrolle über die infizierte Zelle und sie beginnt,
weitere Viren zu produzieren.

Wie bei der Zellteilung werden auch bei der Produktion von Viren durch
infizierte Zellen die Eigenschaften des Virus, das die Zelle befallen hat,
im Regelfall unverändert vererbt, können sich aber durch Mutation ändern.
